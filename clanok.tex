\documentclass[10pt,twoside,english,a4paper]{article}

\usepackage[english]{babel}
\usepackage[IL2]{fontenc}
\usepackage[utf8]{inputenc}
\usepackage{graphicx}
\usepackage{url}
\usepackage{hyperref}
\usepackage{cite}
\usepackage{times}
\usepackage{indentfirst}

\pagestyle{headings}

\title{The influence of video games on people and the human psyche\thanks{Semester project in the subject Methods of engineering work, ac. year 2022/23, management: Ing. Igor Stupavsky}}

\author{Erik Roganský\\[2pt]
	{\small Slovak University of Technology in Bratislava}\\
	{\small Faculty of Informatics and Information Technologies}\\
	{\small \texttt{xrogansky@stuba.sk}}
	}

\date{\small October 11, 2022}


\begin{document}

\maketitle

\begin{abstract}
\ldots
\end{abstract}



\section{Introduction}
Over the last few, video games have started to gain popularity, and the increase is far from stopping. In the past, not many people could afford to play video games. Computers were exceedingly expensive, and gaming consoles were no better. As time passed, electronic devices became more affordable for everyone. The age range of video game players is also expanding precipitously. Video games, in general, were designed to entertain teenagers and young adults. These days, it is not uncommon to find a toddler playing with a smartphone, a mature adult sitting in front of television playing games on a console, or even an elder, playing on a computer. As video games are expanding among more and more people, their influence, whether it is positive or negative, is also getting more vast. That is why we should focus more on the issue of video games and their impacts on humans, which will be the main topic of this work.

\section{The influence of games on the child development} %\label{part1}
\ldots

\section{The influence of games on the adults' behavior}
\ldots

\section{Changes in empathy due to video games}
\ldots

\section{Positive effect on humans}
Growing popularity makes video games an excellent tool that can be used in various positive ways. They have the power to improve the quality of life in an abundance of aspects. Some studies, for example, researched how violent games affect people, and the outcome is remarkable. They found out that games containing violence do not lead to aggression but, on the contrary, improve visuospatial skills. Another study discovered that seniors who regularly or at least occasionally play video games witness better well-being and experience less depression than those who do not~\cite{poz-neg-sol}.


\subsection{Socializing via video games}
As far as society remembers, games played a vital role in socializing. The first game is dated back to ancient times. People played games to meet others and to entertain each other. People created many meaningful bonds through playing games. As time has passed and we have come to a digital era, video games have started to dominate over classical games. In the matter of socializing, MMO (Massively Multiplayer Online) games are one of the best ways to meet people online. They are designed to be social on a massive scale. MMO games are technically a space with hundreds of people that play together, competitively or cooperatively. While meeting people via video games may not be the same as shared experiences, friendships established through playing video games can be similarly meaningful~\cite{poz-neg-sol}. Additionally, people having difficulties establishing normal relationships can also benefit from playing these games because they make it for them to talk through the game than face to face.

\subsection{Education}
Video games have a great tendency to capture people's attention and hold it for long periods of time. On the other hand, learning by itself has the exact opposite effect. These two elements are combined in video game-based learning, which makes it a remarkable learning strategy. But it's important to remember that video games should not be the primary method of education and should only be utilized as a learning aid. There are video games with solely educational purposes, despite the fact that creating such is challenging. Counting, writing, spelling, and pronouncing activities for young children and toddlers are a few examples. Then there are games still being considered educational but aren't designed to teach their players something new, but they help players enhance their already existing abilities. For instance, sudoku helps improve logic, crosswords aid with vocabulary expansion, and Millionaire helps with general knowledge improvement, along with a multitude of other games~\cite{learning}.

\subsection{Ability improvement in other video games}
Even games that are not intended to teach anything can be of benefit. A study shows that almost every video game genre can be beneficial for something. Adventure games are the most common illustration. A compelling plot, action, problem-solving, riddles, puzzles, and many kinds of seeking for things are nearly always included~\cite{learning}. All these things by themselves can improve specific abilities. For instance, the action in adventure games enhances quick reactions (see Figure 1). Strategy games, on the other side, improve one's decision-making skills and thinking abilities. There are a handful of video game genres that can help with improving something, but that is a story for another time. 

\begin{figure}[h]
\centering
\includegraphics[scale=0.16]{adventure}
\caption{A diagram of how adventure games help improve certain abilities}
\end{figure}

\subsection{Foreign language learning}
Despite the fact that video games are still not as successful in the classroom as they could be, they may be significantly beneficial for language acquisition. Numerous studies back up the concept of learning foreign languages through video games but ideally in conjunction with conventional learning methods. Video games' ability to engross their players has been found to be an excellent tool for language learning. They contain endless dialogues in oral and written form (see Figure 2). It is easier to understand the words and sentences since players have substantial control over their in-game actions, decisions, and dialogue choices~\cite{language}. Even the possibilities to pause the game, repeat certain moments, or select alternate options are advantageous. Video games give more vocabulary repetition than movies or books do. For instance, in mini-battles, shooting, solving puzzles, or even in the game menu~\cite{language}. As is well known, repetition is one of the most effective methods for language learning.

\begin{figure}[h]
\centering
\includegraphics[scale=0.16]{language}
\caption{A diagram of in-game language learning process}
\end{figure}

\section{Negative effects on humans}
\ldots

\subsection{Addiction}
\ldots

\subsection{Obesity}
\ldots

\section{Conclusion} \label{conclusion}
\ldots

\section*{Citations} \label{cit}
~\cite{school,behavior,empathy,problems,poz-neg-sol,language,learning}


\bibliography{literatura}
\bibliographystyle{acm}
\end{document}
